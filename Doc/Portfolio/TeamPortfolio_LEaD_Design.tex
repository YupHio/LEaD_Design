\documentclass[12pt]{article}
\usepackage[paperwidth=8.5in,paperheight=11in,margin=1in]{geometry}
\usepackage{float}
\usepackage{lipsum}
\usepackage{parskip}
\usepackage{bbding}
\usepackage{amssymb}
\usepackage{titlesec} 
\usepackage{graphicx}
\usepackage{hyperref}
\usepackage{setspace}
\usepackage[normalem]{ulem}
\usepackage[section]{placeins}
\usepackage[toc,page]{appendix}
\newcounter{subsubsubsection}[subsubsection]
\newcommand{\tline}{\hspace{-2.3pt}$\bullet$ \hspace{5pt}}
\hypersetup{colorlinks=true, linkcolor=black, urlcolor=blue}
\setlength{\parindent}{15pt} % Indent paragraphs (automatically)
\usepackage{pdfpages, caption}


\makeatother
\makeatletter
\setlength{\@fptop}{0pt}

\newcommand\tab[1][1cm]{\hspace*{#1}}

\definecolor{myRed}{RGB}{248, 0, 0}
\definecolor{myGreen}{RGB}{0, 208, 0} %full green too light -P
\definecolor{myBlue}{RGB}{0, 0, 248}

\begin{document}
	\begin{titlepage}
		\centering	
%		\vspace{.25cm}
    
    \begin{figure}[h]
      \centering
%      \includegraphics[width=0.45\linewidth]{icon for team goes here}
    \end{figure} 
  
  {\huge\bfseries \textcolor{myRed}{L}\textcolor{myGreen}{E}a\textcolor{myBlue}{D} Design: \\ Team Portfolio\par}
    
    \title{}
    \date{\vspace{-5ex}} %blank date -P
    \author{%
    	\makebox[.3\linewidth]{\Large\itshape Adrian Beehner}\\Budget Director\\
    	\and \makebox[.3\linewidth]{\Large\itshape Andrew Butler}\\Documenter\\
    	\and \\ \makebox[.3\linewidth]{\Large\itshape Kevin Dorscher}\\Client Liaison\\
    	\and \\ \makebox[.3\linewidth]{\Large\itshape Paul Martin}\\Designer\\
    }
    \let\newpage\relax\maketitle %don't make a new page -P
    \maketitle		
    
    \vspace{4cm} 
    
    {\scshape\Large 
      CS 480/481: Fall 2017 - Spring 2018 \\
      Senior Capstone Design Project \\ 
      UI CS - Wireless Tower of Lights \\
      Sponsor - Dr. Robert Rinker
      \par}
    
     \vspace{4cm} 
    
    \begin{figure}[h]
      \centering
      \includegraphics[width=0.7\linewidth]{assets/uislogan.png}
    \end{figure} 
  
		\vfill		
	\end{titlepage}

	\tableofcontents
	\newpage
	
	\section{Introduction}
	
		\subsection{Project Summary}
		The University of Idaho has, for several years, done various projects involving the Tower of Lights Show and equipping the marching band with light-up glasses. The current "TowerLights" product involves LED-based light bars that are placed in front of front-facing widows of a large buildling (Theophilus Tower) and are then illuminated to play animations alongside/synchronously with music. The goal is to enhance the current "TowerLights" product. The current implementation of this product uses the ethernet wiring system in the building to control the LEDs. The goal of the project described in this document is to convert this part of the system to a wireless operation. This in turn requires the development of a wireless module that would be attached to each of the light bars. Thus this module has to sleep and wake up, as well as respond to wireless signals from a computer, and since it's wireless, these modules will need to be battery powered. Battery power must also be conserved by staying in the sleep state until needed. The purpose of this enhancement is to provide a certain level of portability to have "TowerLights" at other locations. \\
		The product will give the user the ability to run a program that reads in .tan files and .wav files, have this program communicate with a XBee Wireless module on an Arduino that is attached to a computer via USB, then communicate wirelessly with each Arduino receiver, that is battery powered. Each of these Arduino receivers are attached to an LED board, that will then communicate with each LED on that board through wired communication from the Arduino (same one that holds the receiver) to the LEDs. The program that broadcasts the shows will be available for OSX, Windows, and Linux based operating systems.\\
		This documentation lives at \url{https://github.com/YupHio/LEaD_Design/tree/master/Doc/TeamPortfolio_LEaD_Design.tex} \\
		The code for the project can be found at \url{https://github.com/YupHio/LEaD_Design/tree/master/Code}
		
		\subsection{Document Purpose}
	 		This document is a team portfolio for the Fall 2017-Spring 2018 CS 480/481: Senior Capstone Design project at the University of Idaho. The purpose of this document is to outline the methodology, design, and keep a record of this project. It defines terms used, outlines the scope of the project, details specific design choices, meeting minutes, project learning, design goals, specification and constraints, system diagrams, analysis of alternatives, engineering modeling, manufacturing/assembly plan, experimental design, data analysis, balance sheet, and other items.
		
		\subsection{Definition of Terms}
			\begin{itemize}
				\item \textbf{Arduino} - open source computer hardware and software company, project, and user community that designs and manufactures 		single-board microcontrollers and microcontroller kits for building digital devices and interactive objects that can sense and control objects in the physical world\\ (https://en.wikipedia.org/wiki/Arduino)
				\item \textbf{Arduino Shield} - Shields are boards that can be plugged on top of the Arduino PCB extending its capabilities. The different shields follow the same philosophy as the original toolkit: they are easy to mount, and cheap to produce.\\ (https://www.arduino.cc/en/Main/ArduinoShields)
				\item \textbf{Xbee} - The Arduino Xbee shield allows multiple Arduino boards to communicate wirelessly over distances up to 100 feet (indoors) or 300 feet (outdoors) using the Maxstream Xbee Zigbee module.\\ (https://www.arduino.cc/en/Main/ArduinoShields)		
			\end{itemize}
		
		\subsubsection{Arduino IDE}
		The Arduino Integrated Development Environment - or Arduino Software (IDE) - contains a text editor for writing code, a message area, a text console, a toolbar with buttons for common functions and a series of menus. It connects to the Arduino and Genuino hardware to upload programs and communicate with them. \url{https://www.arduino.cc/en/Main/Software}
		
		\subsubsection{Pulse}
		 PulseAudio is a sound system for POSIX OSes, meaning that it is a proxy for your sound applications. It allows you to do advanced operations on your sound data as it passes between your application and your hardware. Things like transferring the audio to a different machine, changing the sample format or channel count and mixing several sounds into one are easily achieved using a sound server. \url{https://www.freedesktop.org/wiki/Software/PulseAudio/l}
		
	\newpage

\section{Team Meetings and Minutes}
	Weekly action items and summaries of progress made are detailed below. Furthermore, subsections discuss what was helpful and what was not during these meetings. Discussion of attendance and participation, as well as contribution and discussion topics are discussed below.

	\subsection{9/14/2017 Team Meeting 1 Notes}
\noindent
Meeting started at 3:30 in room 133 of the library. Adrian's having Internet issues and is unable to attend. 

\noindent
Project priorities: essentially, everything GUI related should be left until the end. Figuring out, prototyping, and testing hardware is the most important thing this semester.

\noindent
We need to schedule another meeting with Rinker to go over more hardware details. We should create some rough drafts of high level UML diagrams before then, and make sure that we have the big picture correct.

\noindent
What resources (old code, parts, lab space, etc) do we have available right now? 

\noindent
See timeline below

\noindent
Meeting adjourned at 4:16, summary and schedule sent to Adrian afterwards

	\begin{figure}[!htb]
		\centering
		\includegraphics[width=140mm]{assets/9-14_Project_Agenda_1.jpg}
		\caption{9/14 Meeting Project Schedule \label{overflow}}
	\end{figure}

	\begin{figure}[!htb]
		\centering
		\includegraphics[width=140mm]{assets/9-14_Project_Agenda_2.jpg}
		\caption{9/14 Meeting Project Schedule \label{overflow}}
	\end{figure}
	
	\clearpage

	\subsection{10/05/2017 Team Meeting 2 Notes}

		\noindent
		Meeting started at 3:30, all members except Adrian preset. Adrian’s running a few minutes late due to construction.

		\noindent
		Diagram status and breakdown (Kevin, started 3:31):

		\noindent
		\begin{itemize}
			\item Need to breakdown exactly how the wireless protocol/transmitting will work. Might not be an actual diagram. 
				\begin{itemize}
					\item What frequency?
					\item How many bits/s?
					\item How many bits/packet?
					\item Order/encoding of packets
				\end{itemize}
			\item Breakdown the LEDs
				\begin{itemize}
					\item How many?
					\item Series or parallel?
					\item Total/individual amperage? (~270 mili-amps per currently)
					\item Voltage drop per diode for each color
				\end{itemize}
			\item Battery specifications
				\begin{itemize}
					\item Chemistry/options/alternatives
					\item Voltage
					\item Amp hours/capacity
				\end{itemize}
			\item Receiver/Arduino board
				\begin{itemize}
					\item What code runs on the Atmega328p chips?
					\item LED driver circuit
				\end{itemize}

		\end{itemize}

		\noindent
		Action items (directly off of the diagram status, ~3:48)
		\begin{itemize}
			\item Wireless description (Paul)
			\item LEDs description (Andrew
			\item Battery specification (Adrian)
			\item Receiver/Arduino board specification (Kevin)
		\end{itemize}

		\noindent
		Meeting adjourned at 4:10
	
	\subsection{11/02/2017 Team Meeting 3 Notes}
	Meeting started at 3:29 with all members present, Adrian via Zoom meeting

	\noindent
	Wiki page: Paul’s gotten it started, pages are created and have some content.\\
	
	\noindent
	Meeting tomorrow with Prof. Rinker: allocate a couple hours. The plan is to discuss circuit details, 		assemble hardware to start testing basic functionality.\\ 
	
	\noindent
	Hardware list(Adrian): List has been created, can hopefully be finalized and whatever new parts are 		needed 	to create a prototype can be ordered.\\
	
	\noindent
	Arduino low power mode: 2 interrupt pins, potentially have to be woken up on low level trigger? Look 		more at datasheet and test Arduino this weekend or early next week.\\
	
	\noindent
	Update portfolio: include everything from snapshot day, update timeline, part list/price per unit if we 	can get details from Rinker on time.\\
\begin{itemize}
\item Wireless design spec and wiki page information in portfolio (Paul)
\item Arduino/Receiver design spec and client meeting section in portfolio (Kevin)
\item  Budget decisions, part list, battery specification in portfolio (Adrian)
\item LED design specification, timeline, and team meeting section in portfolio (Andrew)
\item Finish wiki page (Paul)\\
\end{itemize}

Meeting adjourned at 4:10

\clearpage

	\subsection{11/16/2017 Team Meeting 4 Notes}
	The Team meeting started 3:30 with all members present, Adrian via Zoom meeting.
	
	\noindent
	The team reviews the slide deck created for the design review one presentation. After all team members agree the design review one presentation slides look good we begin to practice our presentation for design review one.
	
	\noindent
	Adrian (on zoom meeting) agrees to be the mock audience for the practice presentations.
	
	\noindent
	The team practices the presentation three full times from start to finish, and 			after some small discussion, all team members decide we are ready for the design 		review one presentation. 
	
	Meeting ajourned at 4:21

\clearpage

	\subsection{01/18/2018 Team Meeting 5 Notes}
	The team meeting started at 3:30 with all team members present, Adrian via Zoom 		meeting.
	
	\noindent
	The main points of this team meeting are as follows:
	\begin{itemize}
	\item Finalize Hardware Decisions

		\begin{itemize}
		\item Select Battery Type
		\item Calculate Expected Battery Life
		\item Determine Resistor Values
		\item Finalize Circuit Layout
		\end{itemize}
		
	\item Begin Initial Prototyping
	
		\begin{itemize}
		\item Prepare for Production of Circuits
		\item Gather all required materials
		\item Prepare for a single wireless light bar prototype
		\end{itemize}
	
	\end{itemize}
	
	\noindent
	The team decides to use a 9v Li Battery, the team also decides to use the existing "Goofy Glasses" circuit as our bast design. Our team also decides to use Series connection for our LED's. 
	
	\noindent
	The team contacts Dr. Rinker to begin requesting parts to begin the construction of an initial prototype.
	
	\clearpage
	
	\subsection{02/15/2018 Team Meeting 6 Notes}
	The meeting began at 3:30 with all team members except Adrian present.
	
	\noindent
	As Adrian is a remote student, and this team meeting was solely for practicing 			the design review two presentation, Adrian did not attend this team meeting, as 		agreed on by the team.
	
	\noindent
	As stated above, the purpose of this team meeting was to practice going over our 		updated design review two slides, as well as practicing our teams design review 		two presentation.
	
	\noindent
	After the team's review of the design review two slides, and a few minor 				alterations to the slide deck, our team begins practice runs of the design review 	two presentation.
	
	\noindent
	After three full practice runs, from start to finish, the team agrees that we are 	well prepared for deign review two.
	
	\clearpage
	
	\subsection{03/01/2018 Team Meeting 7 Notes}
	The meeting began at 3:30 with all team members present, Adrian via Zoom meeting.
	
	\noindent
	The purpose of this team meeting is to discuss, and delegate items that need to 		be updated for the upcoming Snapshot Day 3 presentation. Our team also used this 		meeting time to look into team portfolio additions, and assign team members to 			all items that need to be added into the team portfolio.
	
	\noindent
	The Snapshot Day 3 items to be updated are as follows:
	\begin{itemize}
	\item Update Team Info. (Adrian)
	\item Update Problem Statement (Kevin)
	\item Update General Specifications (Kevin)
	\item Update Project Learning (Andrew)
	\item Update Proof of Design (Adrian)
	\item Update Visualization of Final Product (Paul)
	\item Update Project Completion (Andrew)
	\item Issues and Plans (Paul)
	\end{itemize}
	
	\noindent
	\\The Team Portfolio Additions are as follows:
	\begin{itemize}
	\item Update Battery Specifications (Adrian)
	\item Update LED specifications (Andrew)
	\item Update Team Meetings (Kevin)
	\item Update Client Meetings (Kevin)
	\item Update Prototype Progress (Paul)
	\item Add Design review 1 section (Kevin)
	\item Add Design Review 2 section (Kevin)
	\item Update Wiki Page info. (Paul)
	\item Update Budget and parts list (Adrian)
	\end{itemize}

	\noindent
	\\The team agrees to have all items above completed by Monday 03/05/2018 before 			Snapshot Day 3.

	\clearpage

\section{Client Meetings and Minutes}

	\subsection{9/08/2017 Client Meeting 1 Notes - With Dr.Rinker}

	Meeting started at 3:31, all members present, Adrian on Zoom meeting

Question and Answer with Rinker:

General schedule: Rinker's in CDA start of the week, always in Moscow on Friday, in between depends on events.

Current system in the tower: 3 high powered LEDs (in series) in each room facing the proper direction, controlled over CAT cable from the basement. LEDs prefer constant current over constant voltage, using constant current power source. Each color takes 270 mili-amps. Constant current circuit used here.

Future objectives: Convert system to be wireless. Nodes will need to sleep for a couple days before the show begins, using low power, and should then be remotely wake-able. Power and LED configurations are up to us.

Current goofy lights: broadcast from laptop to Arduino like board, transmits out to the glasses. Wireless protocol related to Zigby. Not wifi, but 802.15.4 (ad-hoc sensor network). Devices can sleep, wakeup, reconnect to network, etc. Zigby handles errors when reconnecting, etc. We avoid using Zigby as we’re broadcasting in real time, and do not want the error handling. Broadcasts on 2.4ghz, regular wifi frequency. 9v lith-ion batteries are being used in the glasses. LEDs in the glasses are in series. Uses a resistor to deliver the correct voltage. Uses the chip from an Arduino, straight up programmed from the Arduino IDE. Atmega328p. Broadcasts to all glasses i.e. DMX. Uses 16 different channels for groups of lights.

802.15.4 only goes at 250kbps. Might want to reduce each channel to 2 bytes instead of 3?

DMX protocol: Used in theater lights, wired protocol, goes through each light sequentially.

Current code is all available for use, we're going to get that from Rinker and put on GitHub(?)

Mouser.com parts. Superbrightleds.com	

Meeting adjourned at 4:39

	\clearpage
	\subsection{9/22/2017 Client Meeting 2 Notes - With Dr.Rinker}
	All members present, Adrian via Zoom, meeting started at 3:32

We only need to deal with .tan file to hardware, there’s another group redesigning the .tan file creator. They’re finishing up in December. Also may be redesigning the interface for the player?

Current implementation uses xbee to transmit to receiver, then transmits to the light controller serially. 

We can probably use old CSAC space to store hardware, work. This has a soldering station too, along with some goofy glasses and the old tower hardware being stored. 

328p chips are super cheap, could definitely use one of those for each light bar. 

Arduino IDE supports turning off bootloader now, etc, which should make development even easier. 

Current player is Linux specific, supposedly has Mac and Windows equivalent libraries though. Pulseaudio and FTDI. Look into making this cross platform compatible.

Main thread of player sends wav bytes to the audio thread, updates lights once the program reaches the proper time. 

Parts needed: transmitter, shield, USB to serial, receiver chips, light bars themselves, batteries. 

Meeting adjourned at 4:33

	\clearpage
	\subsection{11/03/2017 Client Meeting 3 Notes - With Dr.Rinker}
	
	\noindent
	Meeting started at 3:39, all members present, Adrian via Zoom meeting.\\
	
	\noindent
	Constant current circuit: start with the Goofy glasses circuit, figure out what will work for us, 			modify what we need to. Diagrams using PCBArtist(4pcb.com): Each part has a schematic symbol and a 			“footprint” describing what the actual part looks like.\\

	\noindent
	Receiver (MRF chip) can send signals on the 328P’s interrupt pin! Regulator chips should function no 		matter what battery we choose, within reasonable limits, so we should be able to keep using those.\\ 
	
	\noindent	
	Rinker’s going to send us the current circuit diagram so we’ll have access to parts list, details, etc.\\ 
	
	\noindent
	PCBArtist lets you create the traces for the circuit board, then order the custom board based on your 		output. Print as many per sheet as possible, \$33 for each sheet, for students (60 square inches max).\\
	
	\noindent
	Rinker has some breadboard type shields for Arduino that we could use for prototyping if those would be 	helpful. Could basically just add transistors between the Arduino 3.3v supply and the LEDs to create a 		prototype.\\
	 
	\noindent	
	Xbee transmitter is same between Tower Lights and Goofy Glasses.\\ 	

	\clearpage
	
	\subsection{01/26/2018 Client Meeting 4 Notes - With Dr.Rinker}
	Meeting started at 3:30, with all team members present, Adrian via Zoom meeting.
	
	\noindent
	Client Meeting 4 discussion topics:
	
	\begin{itemize}
	\item Goofy Glasses Circuit design
	\item Goofy Glasses Parts List
	\item Obtaining parts to create a prototype
	\item System Design changes brought to Dr. Rinker
	
		\begin{itemize}
		\item Use of 2 9v Li Batteries Run in Parallel
		\item LED configuration changed to 2 sets of 2 LED's run in series, connected 		in parallel
		\end{itemize}
	\item Project completion schedule and planning
	\end{itemize}
	
	\noindent
	Dr. Rinker liked all of our teams purposed design changes except for the idea of 		running 2 9v Li batteries in parallel.
	
	\noindent
	Dr. Rinker informs our team that running two batteries in parallel can be 				problematic. Dr. Rinker notes that running two batteries in parallel can be 			safely done with connecting the batteries with a resistor to limit cross 				charging.
	
	Meeting adjourned at 4:49.
	
	\clearpage
	\subsection{03/23/2018 Client Meeting 5 Notes - With Dr. Rinker}
	Meeting started at 3:30, with all team members present, Adrian via Zoom meeting.
	
	\noindent
	Client Meeting 5 Discussion topics:\\
		\begin{itemize}
			\item ZIF Socket
			\item Total number of final prototypes (to be delivered)
			\item Goofy Glasses Code\\
		\end{itemize}
		
		\noindent
		During client meeting 5, our team received a ZIF socket (Zero Insertion 				Force). The ZIF socket is the last hardware item our team needs to finish the 		construction of multiple prototype circuit boards, and light bars. \\
		
		\noindent		
		Our team asks Dr. Rinker how many circuit boards and light bars he is 					expecting for out final project delivery.\\
		\\
		\noindent
		Dr. Rinker informs our team that four light bars, and prototype circuits is 			plenty to prove we have a working design for the final product delivery.
		\\\\
		
		\noindent
		Meeting adjourned at 4:49.
		\clearpage
	
\section{Project Learning}
	Technologies used to solve problems are described below. Further discussion of these technologies are left in each section's subsections.

	\subsection{Team Info}
	 A discussion of the various pieces of info that relates to team information is provided below.
	
	% Remove Bullets from item list
	{\renewcommand\labelitemi{}
		% Begin list
		\begin{itemize}
			\item \textbf{Team Name}
			\begin{itemize}
				\item LEaD Design
			\end{itemize}
			\item \textbf{Team Members}
			\begin{itemize}
				\item Adrian Beehner
				\item Andrew Butler
				\item Kevin Dorscher
				\item Paul Martin
			\end{itemize}
			\item \textbf{Sponsor}
			\begin{itemize}
				\item Dr. Robert Rinker
			\end{itemize}
		\end{itemize}
	
		\begin{figure}[!htb]
			\centering
			\includegraphics[width = 70mm, height = 90mm]{assets/Team_Info.png}
			\caption{Team Info \label{overflow}}
		\end{figure}
	
	\subsection{Proof of Design}
		A discussion of the various components is shown below, providing evidence of components working together.
	
		% Remove Bullets from item list
		{\renewcommand\labelitemi{}
			% Begin list
			\begin{itemize}
				\item \textbf{LightBar}
				\begin{itemize}
					\item LightBar designed similar to original "Tower of Lights" one
					\item 1 in x 2 in
					\item Size supports common sizes that are used for PCBs and LEDs
				\end{itemize}
				\item \textbf{LED Driver Circuit}
				\begin{itemize}
					\item Similar to "Goofy Glasses" Circuit
					\item Schematic will be very similar, besides the fact that higher voltage and some other additional items will be added
				\end{itemize}
				\item \textbf{Towerplayer Program}
				\begin{itemize}
					\item Modified from various files from original "Tower Player" programs:
					\begin{itemize}
						\item towerarduino.ino
						\item towerplayer.cpp
						\item yswavfile.cpp
						\item yswavfile.h
					\end{itemize}
				\end{itemize}
				\item \textbf{LED}
				\begin{itemize}
					\item LEDs already function on "Goofy Glasses"
					\item Similar design, with battery and circuit providing the power and data to correctly display specific color for LED
					\item Layout of LEDs will actually follow similar design as the original "Tower of Lights" LightBar.
				\end{itemize}
				\item \textbf{Battery}
				\begin{itemize}
					\item 9V Lithium Ion Battery already working on Goofy Glasses
					\item Currently provides 30 minutes of run time
					\item Current Battery choices are between 9V Lithium Ion and 18650 (which would last longer)
				\end{itemize}
			\end{itemize}
		
		A diagram that correlates to the information that is provided above discussing the proof of design id shown below. Images are provided in the diagram to help provide a visual for certain aspects.
	
		\begin{figure}[!htb]
			\centering
			\includegraphics[width = 120mm, height = 140mm]{assets/Proof_Of_Design.png}
			\caption{Proof of Design \label{overflow}}
		\end{figure}

	% Create new Page (NEED TO USE clearpage because we have pictures that will affect it!)
	\clearpage
	
	\subsection{Problem Statement}
	The Wireless Tower Lights problem statement has been updated for Snapshot Day 2.
	
		\begin{figure}[!htb]
			\centering
			\includegraphics[width = 120mm, height = 140mm]{assets/2_Problem_Statement_Update.pdf}
			\caption{Problem Statement Update \label{overflow}}
		\end{figure}
		\clearpage

	
	\subsection{General Specifications}
	The General Specifications have been adjusted to more accurately reflect the design choices and concept.		
		\begin{figure}[!htb]
			\centering
			\includegraphics[width = 120mm, height = 140mm]{assets/3_General_Specifications.pdf}
			\caption{General Specification Adjustments \label{overflow}}
		\end{figure}
		\clearpage

	\subsection{Final Product Vision}
		\begin{figure}[ht!]
			\centering
			\includegraphics[width=170mm]{assets/Lightbar_Model.jpg}
			\caption{3D Model of Final Product \label{overflow}}
		\end{figure}
		\clearpage
		
	\subsection{Design Review 1 Presentation}
	Our teams design review 1 was held on 11/16/2017, in EP 205, at 4:00pm.
	
	\noindent
	The wireless tower lights design review 1 went very well. The team did a few 			practice runs and the presentation went smoothly as a result.
	
	\noindent
	All design choices purposed by our team were accepted by our client (Dr. Rinker), 	and our lead instructor (Bruce Bolden). Both our client and lead instructor noted 	that we understand the problems our team is faced with, and have come up with 			reasonable efficient solutions to solve the problems that our project has 				presented.
	
	\noindent
	Some notes mentioned by our client, Dr. Rinker, that our team need to take into 		consideration include:
	\begin{itemize}
	\item Numbering our Design Review Presentation slides
	\item The consideration of an external clock
	\item Refrain from using the term Zigby (use 802.15.4 instead)
	\item Refrain from using the term Ethernet (use Cat-5 instead)
	\item Think sbout how to divide date to support more software channels than 40
	\end{itemize}

	\clearpage
	
	\subsection{Design Review 2 Presentation}
	Our teams design review 2 was held on , in EP 205, at 4:30pm.
	
	\noindent
	The wireless tower lights design review 2 presentation went well. Our team did a 		few practice runs of the presentation, and as a result the design review two 			presentation went smoothly.
	
	\noindent
	We purposed a few major design changes during the design review two presentation, 	the items can be found below.
	
	\noindent
	Our team purposed the idea of using two 9v Li batteries in parallel, and a 				solution to the issue of cross charging (the reason this idea was originally 			disputed). Our team ides is to run two 9v Li batteries in parallel with a single 		diode connected to the positive lead of each battery. Using diodes, which only 			allow current to flow in one direction, both of our batteries would never be able 	to charge the other, if one battery is producing more voltage then the other. 
	
	\noindent
	Out team also purposed a change to the configuration of our LED's. We presented 		the idea that we plan to run 2 sets of 2 LED's in series, connected in parallel. 		This would give each light bar 4 high powered LED's instead of three, making each 	individual much brighter.
	
	\noindent
	Our team also spoke with Dr. Rinker after the presentation and took down some of 		his suggestions:
	\begin{itemize}
	\item using two hardware channels instead of one.
	\item Using a smaller packet size, with slightly less color resolution to support 	more software channels.
	\end{itemize}
	
	\noindent
	After speaking to Dr. Rinker about the suggestions above, or team has decided to 		increase our software channel support by:
	\begin{itemize}
	\item Cut total packet size to 12 bytes (Down from 24 bytes)
	\item Use two hardware channels (Zigby channels 25 and 26)
	\end{itemize}
	
	\noindent
	Using the 2 changed above the team should be able to support more channels then 		the current 32. Cutting the packet size in half will result in slightly less 			color resolution (16 million colors, down from 32 million colors), but it will 			allow us to create 64 software channels over a single Zigby channel. With the use 	of 2 X-BEE transmitters one transmitter on zigby channel 25, and one transmitter 		on zigby channel 26, will allow our existing software to support 128 channels (up 	from 32).
	
	\clearpage
	
	\subsection{Senior Expo}
	Our team attended the University of Idaho Engineering Expo on 04/27/2018.\\
	
	\noindent
	This was a valuable experience for our team. We were able to view all of the 			College of Engineering senior projects. Our team also had a great opportunity to 		speak with engineers from many different fields, and companies.\\
	
	\noindent
	During the senior design expo, our team was able to speak with multiple different 	groups (expo judges, engineers, high school students, ect.) about the Wireless 			Tower Lights project. We were able to explain the project concepts, design 				choices, and solution clearly to all parties that had any questions. The senior 		expo was a great success, for our team and project.\\ 
	  
	\noindent
	Our team also had a surprising amount of feedback from our senior expo visitors. 		We were able to collect notes on different ideas, and solutions pertaining to our 	project. These suggestions may be of great help to any future Wireless Tower 			Lights team members, as this will not be the last time the Wireless Tower Lights 		will be offered as a senior project at the University of Idaho. 	
	
	\clearpage
	
\section{Design Goals}
	Client needs and project goals are discussed below. A Timeline for these is also included. Discussion of revision of goals, and addition of any new goals is also discussed below.
	
	\subsection{Client Needs}
	The needs of the Client (University of Idaho) are as follows:
		
		\begin{itemize}
			\item LED Light Bars
			\item Microprocessor communication (Arduino)
			\item LEDs bright enough to be a coherent display, visible from a distance in the dark
			\item Wireless Protocol SPI
			\item Battery powered
			\item Receiver Module for Arduino (802.15.14 chip) 
			\item AdrProcessor for designing chip
			\item Low power mode (sleep mode)
			\item Wake up remotely
			\item 1-bit for each color for each window
			\item 802.15.4 protocol, channels 3 bytes (1 for each color, RGB)
			\item Avoid wifi (we don't want to have interference)
			\item Design module
			\item Expand channels (for expanding bandwidth)
			\item 15-20 stories, need to support enough windows
			\item WAV file support
			\item OSX, Windows, and Linux support (Cross-Platform)
			\item .tan file support - 
		\end{itemize}
	
	\subsection{Project Goal}
	The goal of the project is to the extend the versatility of the Tower Of Lights project, which at the moment, gives the user the ability to run a program which reads in a .tan file (animation files for the lights) and .wav files. Then this
	program communicates with a Arduino via Ethernet. Now, the Arduino communicates to each of the LEDs, and tells them which
	color and brightness to be, from the .tan file (thus it basically reads in animation info). 
	The enhancement of the project involves providing cross-platform support, which means having to rework some of the TowerPlayer code so it doesn't use the Pulse library (which is Linux-specific). Also, the enhancement requires making the wired connection to the Arduinos on the LED bar to wireless, this is accomplished by having an Arduino Receiver on each LED Board that receives info sent out from the Arduino connected to the main computer running the program, that Arduino has a XBee Shield attached, which is a wireless module to transmit the info to each Arduino on a board. The Arduino now requires a portable power supply, which needs to be a 9V battery for each Arduino on an LED Board. The final enhancement is that since the LED Boards are running off battery, they require some kind of sleep mode, where they will still be able to receive info (so they can wake up).\\
	
	The product will give the user the ability to run a program that reads in .tan files and .wav files, have this program communicate with a XBee Wireless module on an Arduino that is attached to a Computer via USB, then communicate wirelessly with each battery powered Arduino receiver, on each LED board, that will then communicate with each LED on that board through wired communication from the Arduino (same one that holds the receiver)to the LEDs. The program that runs through this procedure will be available for the OSX, Windows, and Linux based operating systems.

	\subsection{Timeline}
	This is the most recent timeline for the Wireless Tower of Lights project\\
	{ \setstretch{2.0}		
		\scalebox{1}{  		
			\begin{tabular}{r |@{\tline} l}  			
				September  & Planning/Adjustments/Finalize Program Flow         \\			
				October & Hardware Decision/Hardware Tinkering (Arduino/Xbee)            \\			
				November & Hardware Implementation and Initial Prototyping\\			
				December & Prototype Product/Unit Testing     \\			
				January & Product Improvement, Evaluation, and Final Product Hardware Decisions \\			
				February & Implementing and Producing Final Hardware\\			
				March  & Hardware Scale Testing, Software Improvements\\			
				April  & Testing       \\			
				May & Ship/Manufacture (Deliver product)         \\			
			\end{tabular}  		
		}  	
	}
	
	\newpage
    
\section{Specifications and Constraints}
	Discussion of client interviews, pictures, measurements, etc. are provided below. Design specifications and constraints are also presented. Reasoning for any constraints is also mentioned.
	
	\subsection{Arduino / Receiver Design Specification}
	
	\begin{figure}[!htb]
		\centering
		\includegraphics[width = 140mm, height = 95mm]{assets/Arduino_Receiver_Diagram.png}
		\caption{Arduino / Receiver Design Specification \label{overflow}}
	\end{figure}
	
	\indent
	Arduino / Receiver Design Specifications
	\begin{itemize}
		\item Multiple Arduino Atmega 328P boards fitted with a shield and   						  attached receiver chip 
		\item Programming of the individual Arduino Atmega 328P boards using the 					  Arduino IDE (C++)
		\item Receiver chip will delegate the sleep or wake-up modes for each 					  individual light bar
		\item Receiver will also handle input from the transmitting X-Bee, and output 			  data to the LED driver circuit
		\item Creation of the LED Driver circuit which will modify voltage as 						  requested by each set of LED’s to provide a constant current power flow
		\item After modifying voltage accordingly, the LED driver circuit will output 			  the data stream from the receiver to the network that each Arduino 					  Atmega 328P is connected to
	\end{itemize}

	\subsection{Battery Design Specification}
	The list below discusses the attributes for the battery specification, including the requirements, battery chemistry, voltage/capacity, and options/alternatives. Figures "9V Battery Design Specification" and "18650 Lithium Ion Battery Design Specification" corresponding to this information is also shown below.
	
		% Remove Bullets from item list
		{\renewcommand\labelitemi{}
			% Begin list
			\begin{itemize}
				\item \textbf{Requirements}
					\begin{itemize}
						\item Battery required to power the LightBar for TowerOfLights
						\item 3 LEDS on LightBar requires 800 mA
						\item Voltage must be within the range of 8.6 – 9.3 V (Charge)
						\item 10.5 V to run 3 LEDs in a series
						\item 7V for 2 LEDs in a series
						\item Microprocessor based wireless Module distributes the power supply to LEDs on each board
					\end{itemize}
				\item \textbf{Chemistry}
					\begin{itemize}
						\item \textbf{Lithium Ion:} rechargeable battery type, due to high energy density, tiny memory effect, and low self-discharged, lithium ions move from negative electrode during discharge, and back when charging
						\item \textbf{Alkaline:} Popular primary battery (non-rechargeable), dependent on reaction between zinc and manganese dioxide
					\end{itemize}
				\item \textbf{Voltage/Capacity}
					\begin{itemize}
						\item Each LED requires around 3.5 V and each color takes 270 mA
						\item A 9 V battery could support two LEDs in a series, 9V batteries support a wide range of mAh, generally from 400-700 mAh
						\item A 18650 Battery, which has 3.7 V, can be placed in a 18650 holder for 3 batteries, providing 11.1 V, enough to power 3 LEDs in a series (current LightBar setup), with 18650 supporting a range of 1600-3600 mAh
					\end{itemize}
				\item \textbf{Options/Alternatives}
					\begin{itemize}
						\item \textbf{18650 Battery:} large capacity (mAh), allowing LEDs to run longer and can be configured to run LEDs in a series, if making battery pack from these, but requires long charging
						\item \textbf{9V Battery:} Provides smaller capacity, but faster recharge rate. Can only run 2 LEDs for a single 9 V
					\end{itemize}
			\end{itemize}
	
		\begin{figure}[!htb]
			\centering
			\includegraphics[width = 160mm, height = 100mm]{assets/9V_Battery.jpg}
			\caption{9V Battery Design Specification \label{overflow}}
		\end{figure}
		
		\begin{figure}[!htb]
			\centering
			\includegraphics[width = 160mm, height = 100mm]{assets/18650_Battery.jpg}
			\caption{18650 Lithium Ion Battery Design Specification \label{overflow}}
		\end{figure}
		
	% Create new Page (NEED TO USE clearpage because we have pictures that will affect it!)
	\clearpage

	\subsection{LED Design Specification}
	LED specifications and 2 potential solutions detailed below.
	
		% Remove Bullets from item list
		{\renewcommand\labelitemi{}
			% Begin list
			\begin{itemize}
				\item \textbf{Specifications}
					\begin{itemize}
						\item 3 LEDs of each color (red, green, blue) per room
						\item Uses constant current (270-300 mA)
						\item Red LEDs drop ~2.5V per diode
						\item Blue and green LEDs drop ~3.5V per diode
						\item Colors are displayed with pulse-frequency modulation, as each diode can only be fully on or fully off at any moment
					\end{itemize}
				\item \textbf{Circuit Options}
					\begin{itemize}
						\item \textbf{Series:}
							\begin{figure}[!htb]
								\centering
								\includegraphics[height = 100mm]{assets/SeriesLED.jpg}
								\caption{Circuit diagram with LEDs in series \label{overflow}}
							\end{figure}
						\item \textbf{Parallel:}
							\begin{figure}[!htb]
								\centering
								\includegraphics[height = 100mm]{assets/ParallelLED.jpg}
								\caption{Circuit diagram with LEDs in parallel \label{overflow}}
							\end{figure}
					\end{itemize}
			\end{itemize}
	% Create new Page (NEED TO USE clearpage because we have pictures that will affect it!)
	\clearpage

	%Wireless Design Spec
	\subsection{Wireless Design Specification}
	
	% Remove Bullets from item list
	{\renewcommand\labelitemi{}
		% Begin list
		\begin{itemize}
			\item \textbf{Frequency Requirements}
			\begin{itemize}
				\item The wireless protocol needs to have an effective range of potentially up to 100 meters. Additionally, the
				frequency must be one that will work even in crowded venues, with lots of different cellphones, and
				Wi-Fi signals present.
			\end{itemize}
			\item \textbf{Speed Requirements}
			\begin{itemize}
				\item The wireless protocol needs to have the ability to send enough data fast enough to keep up with the
				Tower Lights show. Depending on the total number of light-bars, this number can change. The speed
				requirement will also depend on how many possible colors we implement and how many frames per
				second we will display.
			\end{itemize}
			\item \textbf{Packet Requirements}
			\begin{itemize}
				\item The information packets sent over the wireless protocol must contain all the information needed to set
				the individual light bars to the appropriate color. There can be only one packet that will be sent to all
				the light bars, and each light bar will be encoded with which part of the packet to read.
			\end{itemize}
			\item \textbf{Potential Solution}
			%\begin{itemize}
				%\item Zigbee Protocol
				\begin{figure}[!htb]
					\centering
					\includegraphics[height = 110mm]{assets/Zigbee.png}
					\caption{ \textbf{Zigbee Wireless Protocol} Uses Channels Above Wi-Fi \label{overflow}}
				\end{figure}
			%\end{itemize}
		\end{itemize}
		% Create new Page (NEED TO USE clearpage because we have pictures that will affect it!)
		\clearpage	

\section{System Diagrams}
	Discussion of symbols used, the diagrams themselves, and the software used for the diagrams is discussed below.
	
	\subsection{Current Product}
	The current product flow in regards to the final product is shown below in Figure below. The current setup does not have any battery setup, and requires a wired connection. Changing this is the core of this project, which will improve the versatility of the TowerOfLights product.
	
		\begin{figure}[ht!]
			\centering
			\includegraphics[width=170mm]{assets/What_We_Have.png}
			\caption{Current Product Flow \label{overflow}}
		\end{figure}
	
	
	\subsection{Desired Product}
	The desired product flow is shown in the figure below. The main focus is on the battery that should power each Arduino reciever, as well as the SPI protocol from XBee to the Receiver. This is to make the process wireless instead of wired, which is the main goal of this endeavor.
	
	\begin{figure}[ht!]
		\centering
		\includegraphics[width=170mm]{assets/What_We_Want.png}
		\caption{Desired Product Flow \label{overflow}}
	\end{figure}


	\subsection{PC Running TowerPlayer}
	The diagram for a flow chart depicting the sequence of actions for running the TowerPlayer program on a computer is shown in the figure below. This diagram helps with understanding the underlying software that needs to be setup and used before the hardware can successfully work together.
	
		\begin{figure}[ht!]
			\centering
			\includegraphics[width=170mm]{assets/PCRunningTowerPlayerFlowChartDiagram.png}
			\caption{Flow Chart Diagram for PC Running TowerPlayer \label{overflow}}
		\end{figure}

	% Create new Page (NEED TO USE clearpage because we have pictures that will affect it!)
	\clearpage
	
	\subsection{Lightbar Circuit Schematic}
	Below is the schematic for the custom circuit our team used on each wireless light bar.\\
	
			\begin{figure}[ht!]
			\centering
			\includegraphics[width=170mm, height=145mm]{assets/circuit_schematic.png}
			\caption{Lightbar Circuit Schematic \label{overflow}}
		\end{figure}
		
		\clearpage

\section{Analysis of Alternatives}
	Discussion of possible alternatives and why some alternatives are better is described below. These topics include: safety, moving parts, cost, durability, compatibility, and reliability.
	
	\subsection{9V Battery vs 18650 Battery}
	One of main decisions in this project was deciding whether to use a 9V battery or a 18650 litium ion battery, as they each have advantages and disadvantages. Before desrcibing the choice that was decided upon, its important to examine both batteries, to lay out the benefits and costs of using each in regards to topics such as safety, moving parts, cost, durability, compatibility, and reliability.\\
	
	\noindent\textbf{9V Battery}
			% Remove Bullets from item list
			{\renewcommand\labelitemi{}
				% Begin list
				\begin{itemize}
					\item \textbf{Safety}
					\begin{itemize}
						\item A 9V battery is considered generally safe, as long as handled fairly carefully, but in the wrong hands they can be somewhat dangerous, not to mention the chemical make up of them is harmful to the environment.
					\end{itemize}
					\item \textbf{Moving Parts}
					\begin{itemize}
						\item For the 9V Battery the circuit required a volotage of 11V, but that was if the LEDs were in a series, if they were designed in paralled, the voltage would be reduced to 3.3V. Thus this in its own way was a decision that needed to be made use two 9V batteries with higher voltage and more mAh, but at a higher cost and more moving parts, but that is not the exact topic at hand.
					\end{itemize}
					\item \textbf{Cost}
					\begin{itemize}
						\item The cost of a 9V battery can range drastically, espeically depending on the chemistry of the battery, however the customer specified that they wanted specifically lithium ion batteries, which narrowed down the focus. However, due to the range of mAh, the prices still factors in consideribly, but once again the customer had a specific range around 600 mAh, thus the cost a 9V Lithium Ion battery is around seven dollars.
					\end{itemize}
					\item \textbf{Durability}
					\begin{itemize}
						\item The reason that 9V batteries are among the most popular batteries is due to their high durability and remarkable lifespan. Most 9V lithium ion batteries have a guarantee of at least 10 years of rechargeable battery life, makign the battery a fairly competetive choice. 
					\end{itemize}
					\item \textbf{Compatiability}
					\begin{itemize}
						\item 9V Batteries are known for their wide range of compatability and work easily with a variety of configurations. 9V batteries can easily be utilized within most circuits boards (if they are designed with them in mind)
					\end{itemize}
					\item \textbf{Reliability}
					\begin{itemize}
						\item 9V Batteries are among the most reliable batteries known. It is not difficult to aquire these batteries from almost any store and know that they will provide good quality power (however brands do make a difference). Overall, they are one of the most well-known and used batteries for a reason.
					\end{itemize}
				\end{itemize}
			
	\noindent\textbf{18650 Battery}
		% Remove Bullets from item list
		{\renewcommand\labelitemi{}
			% Begin list
			\begin{itemize}
				\item \textbf{Safety}
				\begin{itemize}
					\item A 18650 battery is considered highly safe as well, perhaps even above the 9V battery counterpart, due to the packaging of the batteries, whihc is usually in builk and with the positve and negatives not easily accessibile. 
				\end{itemize}
				\item \textbf{Moving Parts}
				\begin{itemize}
					\item For the 18650, each one is 3.3V, while the circuit requires at most 11V, so once again if the LEDs were in a series, it would requires three whole 18650 batteries, which are considerably large. This factored into the decision, as more batteries can produce more issues, however even running the LEDs in parallale the voltage would be considered too close to only use one 18650 battery.
				\end{itemize}
				\item \textbf{Cost}
				\begin{itemize}
					\item A single 18650 Battery doesn't range as drastically, as they have only one chemsitry makeup of the battery: lithium ion. Thus instead of having a broad range of selection, the 18650s are quite narrow in the range. The average cost of a single 18650 battery of around 2500 mAh is seven dollars (similar to the 9V), however since it would require 3 batteries, this cost goes up quickly, and the mAh is very far outside the range of the customers needs of 600 mAh, but 18650 batteries all have a mAh range that is generally high
				\end{itemize}
				\item \textbf{Durability}
				\begin{itemize}
					\item The 18650 Batteries are not as durable as their 9V battery counterparts, especially if you buy them in bulk, as they are wrapped with filmsy plastic coverings to hold mutliple batteries together. The bigger issue is, the limited awareness of how durable they are, with such a high mAh generally, its safe to assume they die out at a fairly quick rate (300 recharge cycles and usually replaces within 12-24 months)
				\end{itemize}
				\item \textbf{Compatiability}
				\begin{itemize}
					\item 18650 Batteries are not widely known for being compatible with items (thats not to say they aren't used, as laptop batteries actually use them sometimes and many other items, but they are designed in a particular way for those uses), they are not as widely used as 9V batteries are when it comes to circuit board integration. This can easily be shown by the previous project our product is based off of, Goofy Glasses, which used 9V batteries, not 18650.
				\end{itemize}
				\item \textbf{Reliability}
				\begin{itemize}
					\item 18650 batteries are not exatly known for being reliable. Due to their high mAh rate, they quickly become obselete and need to be replaced. That is not to say they don't provide a great amount of power, as they beat out a lot of similar battieres when it comes to mAh, but these batteries are not easily accesible, as they are not commonly sold in major stores.
				\end{itemize}
			\end{itemize}
	
			\noindent\textbf{Conclusion}\\
			\indent The decision that correlates the best with the current design of the project is the 9V Battery, as with the various costs and benefits mentioned previously about the two, the 9V battery satisfies the client's needs better. THe 18650 would have easily been the better alternative if the product required a much longer operation time, as 9V batteries just don't offer nearly the same amount of mAh as the 18650 batteries do, in the end, the decision is about what is best in meeting the client's needs and making sense in the overall design of the system.
			
	\subsection{9V Battery - 1 vs 2 Quantity}
	The decision betwen utilizing one or two 9V batteries is a difficult one, as the pros and cons of each are heavily important in the decision decision moving froward. it is important to discuss each and determine which one provides the best solution to the problem at hand, while evaluating the other one as a possible alternative.
	
		\noindent\textbf{1 9V Battery}\\
			\indent One 9V Battery woudld follow the decision to have 2-LEDs in a series with one 9V battery, following very closely to the idea of the orignal Goofy Glasses Circuit. This would allow almost no configuraiton, but fails to meet the needs of the client when it comes to having at least three LEDs. But it also avoids the difficult challenge of heavily modifying the Goofy Circuit to work with 18v (as 3 LEDs would requires two 9Vs). This also for a very cheap LightBar that only requires one 9V Battery
			
			\indent The issue with the one 9V Battery is simply that the runtime might just be too short and having only two LEDs in ery limiting. Also, since with very littles configuraitons being made to the circuit this just appears as Goofy Glasses 2.0 (which is in some regards the idea, an evolution of TowerOfLights and Goofy Glasses). However, two LEDs is not enough to provide brightness needed, unless some clever workaround were made.\\
		
		
		\noindent\textbf{2 9V Batteries}\\
			\indent Two 9V Batteries would follow with the decision to have 2 sets of LEDs-in-series. Thus this would have two sets of LEDs that are in parallel then, and two 9V battries in parallel. So instead of the design of three LEDs per LightBar, there would be four, allowing each LightBar to be slighlty brighter. This also allows for less modification of the circuit, as dealing with a 9V in a circuit that is currently designed for 9V means not much needs to be tampered with. This also provides a better duration time to run the LEDs, as the drain from running both the Arduino and Reciever will be reduced by half (between the batteries), also this allows for easily modularity, if one needs more capacity, simply add another battery.
			
			\indent Now the issue that two 9V Batteries present is for one, and the big one, the cost, if each LightBar requires now two 9V lithium Ion Batteries, the cost per LightBar has risen dramatatically. Also, there is the issue of meeting the clients needs (however the client has now agreed that this concept is fine as well). Finally, this means having to adjust the LightBar confifuration that had been already predetermined, since there would now be four LEDs instead of three.\\
		
		\noindent\textbf{Conclusion}\\
			\indent The decision to use two 9V Batteries in the system is the best and most logical decision, as it provides the best benefits with reduced costs. The main advantage is having even more brightness and ease of modularity. While the one 9V Battery otpion would easily win if this product was more concerned with mobility (like the Goofy Glasses), as having 2 9V Batteries is going to weight somedown quite a bit. Also, if costs needed to be cut more as well, but the issues that come alongside using just one 9V battery are too great in this product to utulize it, but perhaps in a similar porject, it would be a good alternative, or it could be used here for extensive and relatively cheap testing.

	\newpage

\section{Engineering Model}
	Discussion of the physical, chemical, and biological system modeling. Also discusses modeling criteria, expected accuracy, and pitfalls. Section of modeling software used is present, as well as data needed and how the data was obtained. Lastly a validation scheme for the model is shown.
	
	\subsection{Design Failure Mode Effects Analysis (DFMEA)}
	The DFMEA is shown below
	
	\includepdf[pages={-}, scale = 1, pagecommand={\null\vfill\captionof{figure}{DFMEA}}]{assets/DFMEA_LEaD_Design_PDF.pdf}

	\newpage

\section{Manufacturing/Assembly Plan}
	Discussion of the fabrication need, a flowchart of process oriented projects, a bill of materials, and the estimated manufacturer and delivery time is discussed below.

	\subsection{Bill of Materials}
	The bill of materials (BOM) is shown below.
	
	\begin{figure}[ht!]
		\centering
		\includegraphics[width=140mm]{assets/Bill_of_Materials_Pg1.png}
		\caption{Bill of Matierals}
	\end{figure}

	\begin{figure}[ht!]
		\centering
		\includegraphics[width=140mm]{assets/Bill_of_Materials_Pg2.png}
		\caption{Bill of Matierals}
	\end{figure}

	 \noindent \textbf{Importance}\\
	 \indent The bill of matierals provides a list of all the products components, details about said component, distributor numbers, manufacturer number, name of company, price, and if pertaining to the PCB schematic, its corresponding symbol in it. This bill of materials is very essential as it provides the basic idea of the manufacturing/assembly plan listing out all of the components needed, where to order, them, at what cost, and what their importance is in manufacturing. 
	
	\newpage

\section{Experimental Design}
	The characterization of the purpose of the experiment, model validation, data gaps, and performance measurement are discussed below. Also the details on documentation, instrumentation, and measurements are also described.
	
		\subsection{Prototype Progress}
		We have completed our first Wireless Lightbar prototype, utilizing 4 LEDs. The prototype can be viewed below:
		
		\begin{figure}[ht!]
			\centering
			\includegraphics[width=140mm]{assets/prototype.jpg}
			\caption{Prototype One}
		\end{figure}
  	
	\newpage	
  	
\section{Data Analysis}
	Documentation on statistical tools used, accuracy of data, and experiments shown below. Discussion on confidence is results also discussed below.

	\newpage

\section{Balance Sheet}
Discussion on initial budget, estimated cost for materials, components, labor, and spending plan are all described below.

		\subsection{Budget}
		The initial budget of the project itself has yet to be decided. However, the current plan should be to not go over \$500, as the budget/spending of the project should be kept to a minimum. However, as our client's needs are still somewhat 'to be decided' when it comes to how much production of products will be required. In the following subsections, everything will be with the goal in mind to try to keep costs down when ever possible, while still fulfilling client's needs, safety, environmental concerns, and so on. Budgeting is an important factor in this project, since the product is largely hardware-based (while still having some software involved). Evaluating and surveying alternatives and solutions to purchasing and building components is an essential aspect to proper Budgeting for the project.
		
		\subsection{Hardware Decisions}
		Decisions for hardware budget are discussed in the list of topics below, each item discusses the reasons, costs, and possible alternatives that could have been taken when deciding how to properly use the budget.

		% Remove Bullets from item list
		{\renewcommand\labelitemi{}
			% Begin list
			\begin{itemize}
				\item \textbf{Arduino Atmega 328P Chip}
				One of best microcontrollers available, affordable, and works with Arduino boards. Other microcontrollers are the LPXCpresso Boards and ARM Cortex-M4 Microcontroller, these are also affordable micrcontrollers, but they lack the simplicity and versatility that the 328P chips have. 
				\item \textbf{"MuRF" Chip}
				The MRF24J40MA chip is one of the best priced radio transceiver modules, with support for Zigbee protocol, which is the desired protocol for the project, it was the optimal choice. There not really any specific alternatives out there that cater to the needs of the project. Zigbee protocol was the client's desired wireless personal network protocol, and also as benefit to the project, as Zigbee is considered simpler and less expensive than other wireless personal networks such as Bluetooth and Wi-Fi.
				\item \textbf{Xbee Sheild}
				Required Shield/Board to interface with the Program that will transmit the data to the LightBoards. The price of this hardware is somewhat high, due to XBee brand, but since this unit is only required once, for whole product, client will be providing the unit free of charge, this not affecting the budget. Unless the product changed the whole wireless personal network to something other than Zigbee, then this was the only available hardware. 
				\item \textbf{9V Lithium Ion Battery}
				The choice for the 9V Battery was due to the 3 LEDs in a series requiring 10.5V in a series. The only other practical alternative was the 18650 lithium battery. However when it comes down to size, mAh, and Voltage, the 18650 battery is the better option, however price is also a large factor, and the 9V wins there.
				\item \textbf{18650 Lithium Ion Battery}
				The choice for the 18650 Battery was due to the 3 LEDs in a series requiring 10.5V in a series. The only other practical alternative was the 9V lithium battery. However when it comes down to size, mAh, and Voltage, the 18650 battery is the better option, however price is also a large factor, and the 9V wins there.
				\item \textbf{Common Anode RGB LED}
				This was a fairly simple choice, as there really is only one type of LED that will work properly for the project, as the product requires the standard LEDs with three dinodes for each color, 3.5V for Red and Green, and 2.5V for Red.
				\item \textbf{LightBar Board}
				This was requested by the client, needed a board to house the product, this is fairly cheap, as it it just na board to house the hardware for the product. The materials/quality of materials to build the board at least at the moment are TBD, while a simple wood board will probably suffice. The size of the board is required to be 1 in x 2 in, thus with this estimate the board would be a fair price for housing the hardware, and providing easy assembly (since with wood the parts can be screwed in).	
				\item \textbf{LED Driver Circuit}
				The actual LED Driver Circuit will be designed by the team, thus there won't be an additional cost in designing, as once the design is done, there are companies that allow the consumer to design PCBs (Printed Circuit Boards) and then order as many PCBs as needed. The company that LEaD Design will be working with to order the circuits is 4PCB, a company that "Specializes in printed circuit board manufacturing and PCB assembly, including prototype and production circuit boards" (www.4pcb.com). They offer special discounts to students, where a 60 $in^2$ board can be created for \$33, and the consumer can put any amount of PCBs, as long as they fit within the 60 $in^2$ size (the 60 $in^2$ is not limited to 6 in x 10 in either, it can be any dimension that fits 60 $in^2$). Thus until the design of the LED Driver Circuit is done, the actual cost per unit of it cannot be determined. The goal in the design process will be to try to make the circuit as small as possible while fulfilling the requirements and keeping in mind the manufacturing process of the circuit (if it is too small, it will be difficult to attach the circuit to rest of the product).
			\end{itemize}
			
			\subsection{Hardware List/Cost of Materials}
			The figure below is the current Hardware list, describing the physical components of the product, their required quantity, and cost per unit. These specifications are still subject to change and evaluation.
			
			\begin{figure}[ht!]
				\centering
				\includegraphics[width=170mm]{assets/HardwareList.jpg}
				\caption{Hardware List \label{overflow}}
			\end{figure}
			
			\newpage

\section{Other Items}
	File management, archiving, documenting any issues, reports of accidents/incidents/near misses/precautions are described below.
	
	\subsection{Team Wiki Page}
	Our team has done a lot of work during this project. Some of the highlights of the projects design elements as well as more team information has been added to our Wiki page.
	\\
	\\
	The team Wiki Page for LEaD Design can be viewed at the following URL: \\
	\url{http://mindworks.shoutwiki.com/wiki/Wireless_Tower_of_Lights}
	
	\subsection{LEaD Design Team Contract}
	The team contract for the \textit{LEaD Design} is shown below. The contract discusses the various professional approaches the team will be held accountable to act towards during the time spent on the project. The contract is an important document, as it dicusses the various inner workings of how the team will work on assignments, resolve conflicts, manage work, make decisions and so on. The contract is four pages total.
	
		\includepdf[pages={-}, scale =1, pagecommand={}]{assets/LEaD_Design_Team_Contract.pdf}			% {-} means include all pages
		


\end{document}
